\documentclass{westarticle}
\input{C:/Users/jstudyvin/GoogleDrive/latex/header}
\usepackage{setspace}
\usepackage{enumitem}
\title{Modeling Fatality Distances}
\author{Paul Rabie, Daniel Riser-Espinoza, Jared Studyvin}
\begin{document}
\westtitle
\doublespace
A probability density function (pdf) of fatalities found around turbines is
needed to calculate an area correction factor. The pdf is a function
of the distance of the carcass from the turbine, assuming isotropy.

If all of the carcasses are assumed to be detected with equal
probability the carcass distances can be modeled in a typical
fashion and standard maximum likelihood estimates can be obtained. In
practice the probability of detection is not equal for all carcasses
and in particular the probability of detection is a function of the
distance from the turbine mast.

This document presents some of the ideas to account for differing
probabilities of detection and some of the questions and difficulties
associated with this process. The first section describes the weighted
distribution, as introduced by Dan Dalthorp, with the two versions
depending on the choice of weights  and the second section introduces the weighted
likelihood. This paper does not solve the problem but is a foundation
for further discussion to answer the remaining questions.

\section{Weighted Distribution}

Let $f(x|\theta)$ be some known pdf (e.q. Gamma, Weibull, etc.) with
$\theta$ being the associated vector of parameters. In the context of
carcass fatalities at wind turbines $x$ represents distances from the
turbine.

The weighted distribution (WD) is defined as
\begin{align}
  \label{eq:fstar}
f_j^*(x|\theta) = \frac{w_j(x)f(x|\theta)}{\int_{I_f} w_j(y)f(y|\theta)dy}
\end{align}
where $I_f$ is the support associated with $f(x|\theta)$ and
$w_j(x)$ is the weighting function.


\begin{enumerate}[label=\bf{Q\arabic*:}]
\item \bf{How should $w_j(x)$ be defined?}
\end{enumerate}


We believe that $w_j(x)$ should be 
\begin{align}
  \label{eq:w}
  w_1(x) \propto \mbox{ probability of detection at distance x}
\end{align}
or
\begin{align}
  \label{eq:inversew}
  w_2(x) \propto \frac{1}{\mbox{ probability of detection at distance x}}
\end{align}
%Our initial testing gives more credence to $w(x)$ according to
%Equation \ref{eq:w}.
It should be noted that $w_j(x)$ is a function of
distance and if two carcasses are the same distance from the turbine
the weight is the same for both carcasses.


In either case of the choice of $w_j(x)$, an estimate of $\theta$ is
obtained from maximizing the log likelihood of $f_j^*(x|\theta)$,
\begin{align}
  \label{eq:Lstar}
l_j^*(\theta | \underbar{x})= \sum_{i=1}^{n}log(w_j(x_i)) +
  \sum_{i=1}^{n}log(f(x_i|\theta)) - nlog(\int_{I_f} w_j(y)f(y|\theta)dy)
\end{align}
where $x_i$ is the distance from the turbine for the $i^{th}$
($i=1,2,\cdots,n$) carcass. The result from maximizing
$l_j^*(\theta|\underbar{x})$ (Eq. \ref{eq:Lstar}) produces $\hat\theta_j^*$.


Once the choice of $w_j(x)$ has been made and $\hat\theta_j^*$ obtained,
the question becomes how to apply $\hat\theta_j^*$ to calculate the area
correction. The estimate $\hat\theta_j^*$ can be applied to $f$, $f^*_j$
or
\begin{align}
  \label{eq:fdagger}
f_j^{\dagger}(x|\theta) = \frac{f(x|\theta)}{\int_{I_f} w_j(y)f(y|\theta)dy}
\end{align}


\begin{enumerate}[label=\bf{Q\arabic*:},resume]
  
\item \bf{$\hat\theta_j^*$ can be used to calculate the densities of
    $f_j^*$, $f$ or $f_j^{\dagger}$. Which of the three are meaninful
    in describing either the observed dat or the true underlying
    distribution fo carcasses using the weighting function $w_1(x)$?
    Or using $w_2(x)$?}
% \item \bf{Does $\hat\theta_j$ correspond to $f_j^*(x|\theta)$,
%     $f(x|\theta)$ or $f_j^{\dagger}(x|\theta)$?}
% \item \bf{Should $f_j^*(x|\theta)$, $f(x|\theta)$ or
%     $f_j^{\dagger}(x|\theta)$ be used
%     to calculate the area correction?}
% \item Between the three functions: $f^*(x|\theta)$, $f(x|\theta)$, and
%   $f_j^{\dagger}(x|\theta)$, which corresponds to the observed data
%   and which corresponds to the approximation of the ``truth''?
\end{enumerate}

\section{Weighted Likelihood}

The weighted likelihood (WL) applies the weights to the likelihood and not
the distribution itself. Assuming some distribution $f(x|\theta)$ that
models the distances of carcasses from a turbine, which has
corresponding likelihood $L(\theta|x_i)$.
The weighted likelihood is
\begin{align}
  \label{eq:wlike}
  WL(\theta|\underbar{x}) = \prod_{i=1}^{n}L(\theta|x_i)^{w_i}
\end{align}
where $w_i$ is the weight associated with the $i^{th}$ carcass. Please
note, $w_i$ is not a function but simply a weight value. 
We believe the choice for the weights ($w_i$) should be proportional
to the inverse of the probability of detection. 
This changes the likelihood to a pseudo
increase in sample size. For example if the probability of detection
is one half at distance $x_i$, then $w_i=2$, which is analogous to finding two carcasses
at distance $x_i$.

\begin{enumerate}[label=\bf{Q\arabic*:},resume]
\item \bf{Does the WL allow weights to be unique to each carcass?}
  % Using the WL allows the weights to be unique to every
% carcasses, i.e. two carcasses with the same distance from turbine can
% have different weights based on plot type, for example.

\item \bf{Is the assumption of the WL, that the observed data does not follow
$f(x|\theta)$ but the estimate $\hat\theta$ from Equation
\ref{eq:wlike} does correspond to $f(x|\theta)$?}
% The idea is the
% weights allow $\hat\theta$ to be recovered because unequal probability of
% detection is accounted for by the weights.

\end{enumerate}

\section{Next Step?}

We see three possible choices: the WD with the weights being the inverse probability of
detection (Eq. \ref{eq:inversew}), the WD with the weights being the non-inverse of
probability of detection (Eq. \ref{eq:w}), and the
WL with the weights being the inverse of probability of detection (Eq. \ref{eq:wlike}). The main question is
what should be done next?

\end{document}
